
% Definições para glossario

% ATENCAO o SHARELATEX GERA O GLOSSARIO/LISTAS DE SIGLAS SOMENTE UMA VEZ
% CASO SEJA FEITA ALGUMA ALTERAÇÃO NA LISTA DE SIGLAS OU GLOSSARIO É NECESSARIO UTILIZAR A OPÇÃO :
% "Clear Cached Files" DISPONIVEL NA VISUALIZAÇÃO DOS LOGS 
% ---
% https://www.sharelatex.com/learn/Glossaries

% Normalmente somente as palavras referenciadas são impressas no glossário, portanto é necessário referenciar utilizando :
% \gls{identificação}            
% \Gls{identificação}            
% \glspl{identificação}            
% \Glspl{identificação} 

\newglossaryentry{admin}{
                name={Admin},
                description={Administradores, \gls{enum} que representa os funcionários que receberão essa funcionalidade naquele semestre/ano.} } 

\newglossaryentry{enum}{
                name={enum},
                description={Tipo de classe especial que permite o armazenamento de um grupo de valores constantes.} }

\newglossaryentry{excel}{
                name={Excel},
                description={Editor de outras planilhas produzido pela empresa Microsoft.} }

\newglossaryentry{django}{
                name={Django},
                description={Framework web full stack open source baseado em Python, gratuito e de alto nível, focado em desenvolvimento rápido para web.} }

\newglossaryentry{git}{
                name={Git},
                description={Sistema utilizado principalmente por programadores para hospedar códigos e arquivos versionados pelo Git.} }

\newglossaryentry{github}{
                name={GitHub},
                description={Sistema utilizado para hospedar códigos e arquivos versionados pelo Git. Disponível no endereço \url{<https://github.com/>.}} }

\newglossaryentry{globo}{
                name=Globo,
                description={Rede de televisão comercial brasileira aberta.} }

\newglossaryentry{feedback}{
                name={feedback},
                plural={feedbacks},
                description={Avaliação de uma ação ou de uma pessoa que leva em consideração uma série de fatores sobre o processo almejado; pode ser positiva ou negativa.} }

\newglossaryentry{framework}{
                name={framework},
                plural={frameworks},
                description={Junção de códigos, como bibliotecas, que traz como resultado final uma funcionalidade genérica, a fim de resolver problemas recorrentes e agilizar o desenvolvimento da aplicação.} }

\newglossaryentry{latex}{
                name={LaTeX},
                description={Sistema/programa de marcação de documentos, com alta tipografia; padrão para textos científicos.} }

\newglossaryentry{login}{
                name={login},
                description={Processo de acesso em um sistema através dos dados de identificação de um usuário.} }

\newglossaryentry{microsoft}{
                name={Microsoft},
                description={Uma das maiores empresas de tecnologia criada por Bill Gates e Paul Allen.} }

\newglossaryentry{mozilla}{
                name={Mozilla},
                description={Comunidade global de software livre responsável por criar o navegador Firefox, dentre outros navegadores e aplicações.} }

\newglossaryentry{open source}{
                name=open source,
                description={Código-fonte aberto de um software, adaptável à finalidade desejada.\cite{nascimento_2014} }}

\newglossaryentry{permuta}{
                name={permuta},
                plural={permutas},
                description={Trocas; nesse caso, de aulas.} }

\newglossaryentry{permutação}{
                name={permutação},
                description={Processo de troca; nesse caso, entre dois docentes, seguindo todos critérios de regramento.} }

\newglossaryentry{pinterest}{
                name={Pinterest},
                description={Rede social de compartilhamento de fotos}
}           

\newglossaryentry{professor}{
                name={Professor},
                description={Professor, \gls{enum} que representa o docente no sistema.} }

\newglossaryentry{python}{
                name={Python},
                description={Linguagem de programação de alto nível, utilizada em aplicações desktop, web, servidores e em ciência de dados.} }

\newglossaryentry{superAdmin}{
                name={SuperAdmin},
                description={Administrador Superior, \gls{enum} que representa os funcionários que receberão essa funcionalidade naquele semestre/ano.} }

\newglossaryentry{svn} {
                name={SVN},
                description={Tem como objetivo fazer o controle de versão de arquivos, inicialmente tinha o propósito de ser um substituto melhorado do Sistema de Versões Simultâneas.} }