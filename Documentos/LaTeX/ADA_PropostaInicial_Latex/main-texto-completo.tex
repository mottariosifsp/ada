%% Adaptado a partir de :
%%    abtex2-modelo-trabalho-academico.tex, v-1.9.2 laurocesar
%% para ser um modelo para os trabalhos no IFSP-SPO
\documentclass[
    % -- opções da classe memoir --
    12pt,               % tamanho da fonte
    openright,          % capítulos começam em pág ímpar (insere página vazia caso preciso)
    %twoside,            % para impressão em verso e anverso. Oposto a oneside
    oneside,
    a4paper,            % tamanho do papel. 
    % -- opções da classe abntex2 --schwinn
    % Opções que não devem ser utilizadas na versão final do documento
    %draft,              % para compilar mais rápido, remover na versão final
    paginasA3,  % indica que vai utilizar paginas em A3 
    BIBLATEX,           % indica para utilizar BIBLATEX em vez do abntex2cite
    REFINDENT,          % não fica exatamente no formato da ABNT, mas melhora muito a formatação
                        % não utilizar REFINDENT na versão final
    MODELO,             % indica que é um documento modelo então precisa dos geradores de texto
    TODO,               % indica que deve apresentar lista de pendencias 
    % -- opções do pacote babel --
    english,            % idioma adicional para hifenização
    brazil              % o último idioma é o principal do documento
    ]{ifsp-spo-inf-cemi} % ajustar de acordo com o modelo desejado para o curso

% ---
% Informações de dados para CAPA e FOLHA DE ROSTO
% ---
\titulo{ADA - Atribuição de Aulas}

% Trabalho em Equipe
% ver também https://github.com/abntex/abntex2/wiki/FAQ#como-adicionar-mais-de-um-autor-ao-meu-projeto
\renewcommand{\imprimirautor}{
\begin{tabular}{lr}
ANA PAULA MOURA MESSIAS DE SOUZA & SP3044505 \\
GUSTAVO SANTOS COSTA SOARES & SP3044491 \\
HENRIQUE LUIS BAESA & SP3045048 \\
ISABELLA VALERIO MAZARÁ & SP3045463 \\
JOSINEUDO DAS CHAGAS ARRUDA & SP3045439 \\
PAULO KENJI YOKOTA MUNEISCHI & SP3045382\\
\end{tabular}
}

\disciplina{PDS - Prática para Desenvolvimento de Sistemas}

\preambulo{Projeto apresentado no quarto ano do curso de informática integrado ao ensino médio, do Instituto Federal de Educação, Ciência e Tecnologia de São Paulo - Câmpus São Paulo, para a conclusão da disciplina de Projeto de Desenvolvimento de Sistemas.}

\data{2023}

\renewcommand{\orientadorname}{Professor:}
\orientador{GUSTAVO FORTUNATO PUGA}
\renewcommand{\coorientadorname}{Professor:}
\coorientador{LEONARDO ANDRADE MOTTA DE LIMA}

% ---
% informações do PDF
\makeatletter
\hypersetup{
        %pagebackref=true,
        pdftitle={\@title}, 
        pdfauthor={\@author},
        pdfsubject={\imprimirpreambulo},
        pdfcreator={LaTeX with abnTeX2 using IFSP model},
        pdfkeywords={abnt}{latex}{abntex}{abntex2}{IFSP}{\ifspprefixo}{trabalho acadêmico}, 
        colorlinks=true,            % false: boxed links; true: colored links
        linkcolor=blue,             % color of internal links
        citecolor=blue,             % color of links to bibliography
        filecolor=magenta,              % color of file links
        urlcolor=blue,
        bookmarksdepth=4
}
\makeatother
% --- 

% carregando aqui referencias quando utilizando BIBLATEX
\IfPackageLoaded{biblatex}{%
\addbibresource{referencias.bib}
\addbibresource{exemplos/abntex2-doc-abnt-6023.bib}
}{}

% ----
% Início do documento
% ----
\begin{document}


% Retira espaço extra obsoleto entre as frases.
\frenchspacing 

% ----------------------------------------------------------
% ELEMENTOS PRÉ-TEXTUAIS
% ----------------------------------------------------------
\pretextual

% ---
% Capa
% ---
\imprimircapa

% ---
% Folha de rosto
% (o * indica que haverá a ficha bibliográfica)
% ---
\imprimirfolhaderosto
%\imprimirfolhaderosto*
% ---

% Quando registrado na biblioteca
%\input{pre-fichacatalografica}

%Obrigatório para trabalhos com bancas oficiais
%\input{pre-aprovacao}


% -- resumo obrigatório
\input{pre-resumos}


\pdfbookmark[0]{\listfigurename}{lof}
\listoffigures*
\cleardoublepage
% ---

\input{pre-siglas}

% ---
% inserir o sumario
% ---
\pdfbookmark[0]{\contentsname}{toc}
\tableofcontents*
\cleardoublepage
% ---


% ----------------------------------------------------------
% ELEMENTOS TEXTUAIS
% ----------------------------------------------------------
\textual

% ----------------------------------------------------------
% Introdução
% ----------------------------------------------------------
\chapter[Introdução]{Introdução}

No \ac{ifsp} Câmpus São Paulo, são mais de 350 docentes, com 56 na área de \ac{dit}. Logo, é essa a quantidade que participa da atribuição de aulas semestral e/ou anual, indicando seu extenso e intrincado processo. Além da referida quantidade, o que dá essas características à atribuição são os critérios a serem seguidos, desde a ordem de prioridade dos docentes até as preferências colocadas por cada um no \ac{fpa} e os regramentos presentes na resolução vigente – em 2023, ainda rege a Resolução n°109/2015, de 4 de novembro de 2015. 

Todo esse processo é realizado manualmente, através da entrega do \ac{fpa}, da ferramenta \gls{excel} e da comunicação particular constante entre o administrador daquele ano e o docente, principalmente no caso de \glspl{permuta}. Assim, procedem adversidades, conflitos interpessoais e atrasos, relatados semestralmente pelos docentes, principalmente pelos que ficam no final da fila de prioridade (os substitutos), e comentados, após contato da equipe, pelo administrador das atribuições da \ac{dit} atual, Evandro ..., e antigo, Leonardo Motta, os quais enfatizaram sobre a consequente sobrecarga em suas funções, ao tentar equilibrar a vontade de todos e, ao mesmo tempo, cumprir com a lei. 

Em decorrência disso, surge a necessidade da automatização de parte dos processos, que tem como objetivo a aprimoração do andamento do fluxo de trabalho \cite{sydle_2023}, trazendo como resultado o aumento da produtividade e a redução de custos e de erros \cite{totvs_2022}.

E, à vista do que foi exposto, o projeto retratado propõe a elaboração de um sistema que automatize os principais processos da atribuição - as seleções do \ac{fpa} e a \gls{permutação} - e, simultaneamente, cumpra o exigido na Resolução, e nos outros critérios estabelecidos hoje (como a prioridade da escolha do docente na atribuição) e que podem ser posteriormente adicionados. Esse sistema é o \ac{ada}.


\section{Objetivo}

O \ac{ada} visa apresentar uma solução e uma aprimoração às problemáticas da atribuição de aulas. Logo, oferecer um sistema Web responsivo de Single Application Page (\ac{spa}) aos funcionários, que automatize essa atribuição e a respectiva e consecutiva \gls{permutação} (caso habilitada), sem a necessidade de organização manual e de negociações individuais e extraoficiais. 

Com processos correspondentes aos problemas centrais, terá a implementação do \gls{login}, pelo e-mail oficial do Instituto no Google, tratando do gerenciamento geral dos docentes e dos administradores; da automatização do \ac{fpa}, tratando da dificuldade de estruturação da grade horária seguindo todos os critérios e os regramentos; e da automatização das negociações à \gls{permuta}, tratando dos atritos e da dificuldade de comunicação gerados.

Consequentemente, o sistema proporcionará, a princípio, um ambiente em que o administrador superior e os subadministradores consigam controlar e ordenar os critérios às suas área e subáreas, respectivamente, e habilitar funções como a \gls{permutação} e a desativação de um  em determinada matéria. Ademais, proporcionará um ambiente em que o docente consiga selecionar todas as suas preferências e solicitar suas \glspl{permuta}(caso habilitadas) em um único local, sem demasiadas complicações e processos. 

Uma operação antes com responsabilidades individuais e organização manual, a qual incorre de mais erros devido a subjetividade e os problemas humanos, passará a ser uma operação tecnológica mais limpa e funcional, com menos erros.

\section{Análise de concorrentes}

A Análise de Concorrência é valiosa para obter conhecimento sobre como outros sistemas – com o mesmo propósito ou um próximo – desenvolvem seu projeto, implementam seus processos, atraem clientes, apresentam sua plataforma, gerenciam seus dados, fecham parcerias, entre outros; fatores importantes a serem considerados tanto para o aprimoramento do sistema que você está realizando, quanto para saber com quem está disputando o mercado. 

Nessa pesquisa, a equipe achou algumas concorrências referentes ao processo de atribuição de aulas. Entre elas, vale a pena serem citadas a \ac{sed} e o \ac{sig} .
A \ac{sed} apresenta um sistema que abrange toda a rede educacional estadual de São Paulo (SP) \cite{secretaria_sp}. Através dele, o docente pode manifestar seu interesse em aulas vagas e livres ou em substituição das escolas, pela pesquisa por uma escola específica ou uma disciplina. Quanto à prioridade de escolha, o docente pode alterar a ordem da sua seleção de acordo com as escolas em que prefere lecionar. 

E o \ac{sig} \cite{sig} aparenta apresentar um sistema parecido, onde o docente preenche e envia a inscrição e o requerimento de ampliação de carga horária de forma digital. Entretanto, as informações são escassas, baseadas na página de divulgação da \ac{urh} \cite{urh}, onde está o sistema; o tutorial leva a uma página de erro. Contudo, uma diferença fundamental é o fato de ser voltado apenas às \ac{etecs}.

Com o conhecimento adquirido foi possível observar pontos a serem implementados no \ac{ada}, como a disponibilidade de um tutorial no início e mensagens de ajuda ao longo da página – além do ótimo design da \ac{sed} que pode servir de inspiração. E pontos semelhantes foram justamente essa escolha de aulas pelo docente, de acordo com sua subárea, e o \textit{login} que não permite o cadastro de pessoas não autorizadas (um usa do \ac{cpf} e o outro do e-mail).

Todavia, foi igualmente possível observar a necessidade do sistema apresentado neste projeto, que prevê erros retratados nos concorrentes, como o atraso considerável do primeiro devido às longas filas de seleção, a ausência de verificação de componente curricular, abrindo uma brecha para qualquer docente lecionar a disciplina sem ter o nível de escolaridade necessária, e os erros no processo de pontuação para prioridade de escolha – retratados em uma matéria \cite{g1_2020} da \gls{globo} e relatados pela Inês Paz, coordenadora da subsede da \ac{apeoesp} e vereadora de Mogi, \cite{acioli_2021}: “A classificação saiu com muitos erros e os professores não estão tendo um retorno às suas perguntas, enquanto isso, a atribuição continua acontecendo com esses profissionais correndo o risco de se prejudicarem”. Já do segundo concorrente, a falta de tutorial e dificuldade na compreensão da página de \textit{login}. 

À vista do citado acima, o sistema \ac{ada} subsidia uma série de ações que permitem o cruzamento de dados e processos que diferenciam a ideia de qualquer outra anterior; e, por ser voltado ao Instituto em específico, igualmente permite uma melhor análise dos \glspl{feedback} e tratamento dos erros. Portanto, será implementado de forma a considerar boas práticas de concorrentes e aplicações parecidas, e, principalmente, dificuldades nelas encontradas, não atendo-se a um ciclo de falhas.

%\input{textos-revisao-literatura}

%---------------------------------------------------------------------------------------
\chapter{Ideia}

A fim de implementar um sistema que trate dos problemas citados e consiga atingir os objetivos propostos, são necessários processos e a utilização de determinadas tecnologias, citadas no subtópico . 

Os processos principais são três, o cadastramento dos usuários, a automatização do \ac{fpa} e a possibilidade de habilitação de outros processos, por parte dos administradores, com destaque à permutação dos horários já atribuídos e à desativação de um docente em determinada matéria.

\section{Cadastramento dos usuários}

Preliminar à qualquer utilização do \ac{ada}, o Administrador Superior (\gls{superAdmin}), será cadastrado pelos próprios programadores e terá o maior nível de acesso, podendo realizar quaisquer alterações e controlar quais serão os Administradores (\gls{admin}). 

Então, os outros funcionários receberão um link para acessarem o \ac{ada} via Google, pelo e-mail institucional - o que evita acessos não permitidos, e serão atribuídos instantaneamente ao papel de Professor (\gls{professor}); como mencionado, a mudança desse nível de acesso para o de \gls{admin} é realizada pelo \gls{superAdmin}. E acessos posteriores poderão ser através do Google ou do prontuário e senha.

\subsection{Configuração do ambiente}

A configuração do ambiente é um subprocesso, em que o \gls{superAdmin} será responsável por habilitar a possibilidade de \glspl{permuta} e de desativação do docente em uma disciplina; prazos limites à organização; e definição ou atualização dos critérios da atribuição - baseados na legislação vigente e na ordem de prioridade de escolha das disciplinas.

E o \gls{admin} será responsável pela subárea, consequentemente, por subir a grade horária; determinar prazos específicos; autorizar a \gls{permutação} e se deseja participar da aprovação das \glspl{permuta}; controlar os docentes desativados; e adicionar\footnote{Essa adição será manual e de acordo com a prioridade escolhida. Portanto, um subprocesso, onde o \gls{admin} colocará os docentes na ordem e, igualmente, poderá alterá-la em caso de erro ou modificações futuras.} os que participarão de sua subárea. 


\begin{figure}[h]
    \centering
    \includegraphics[width=0.85\textwidth]{anexos/Fluxograma/FluxogramaCadastramento.png}
    \caption{Fluxograma dos pré-processos de atribuição}
    \label{fig:figura1} 
\end{figure}

\section{Automatização do FPA}

Finalizada a organização do sistema pelos administradores e todos os docentes cadastrados nas subáreas, eles poderão acessar o sistema e iniciar o processo de escolha da disponibilidade de horários e da preferência de aulas (prioritária e secundária) e de atividades. Conforme é realizado esse processo, o ADA verifica se cada escolha segue os regramentos, e impossibilita a escolha de disciplinas em conflito; igualmente, informa com uma mensagem breve caso o docente selecione uma em que foi desativado. 

\begin{figure}[t]
    \centering
    \includegraphics[width=0.7\textwidth]{anexos/CasosDeUso/CasoDeUso_EnvioFPA.png}
    \caption{Caso de uso do envio de preferências}
    \label{fig:figura2} 
\end{figure}

A determinação da preferência de atividades poderá ser modificada dentro do prazo de entrega estabelecido pelo \gls{admin}. Porém, ao encerrar o prazo, o \ac{ada} percorre a lista de docentes, em ordem decrescente, e atribui as aulas de acordo com o selecionado. O processo é interrompido - e é armazenado o que já foi feito - caso haja conflito com uma disciplina já escolhida; assim, aquele docente receberá uma solicitação para alterar sua escolha dentro de determinado prazo.

\begin{figure}[h]
    \centering
    \includegraphics[width=0.7\textwidth]{anexos/Fluxograma/FluxogramaProcessoAtribuicaoAulas.png}
    \caption{Fluxograma da atribuição de aulas}
    \label{fig:figura3}
\end{figure}

\section{Permutação}

A permutação é aberta, caso habilitada com a conclusão da grade pelo sistema. De modo geral, é feita com a solicitação de um docente pela troca de sua aula por uma específica do outro, selecionada na grade. É impossibilitada mais de uma solicitação, ao mesmo tempo, para uma mesma aula; Apenas é liberada quando essa for aceita ou recusada. Igualmente é impossibilidada a solicitação de alguma que descumpra o regramento. 
Caso o \gls{admin} seja moderador, ele terá que aprovar a aceitação da permuta pelo segundo docente.


Por fim, é gerada a grade horária final, onde os docentes e os administradores conseguem visualizar e salvar a atribuição de aulas da subárea. Além da possibilidade de gerar o \ac{fpa} com essa grade pronta.

\begin{figure}[h]
    \centering
    \includegraphics[width=1\textwidth]{anexos/CasosDeUso/CasoDeUso_ProcessoPermutaFULL.png}
    \caption{Caso de uso troca de Aula}
    \label{fig:figura4} 
\end{figure}

\section{Tecnologias e ferramentas aplicadas}

Em vista do desenvolvimento do \ac{ada} de maneira concisa e eficaz, a implementação de tecnologias e suas respectivas ferramentas se faz necessária. Além disso, repositórios de controle de versão e Integrated Development Environment (\ac{ide}) deverão, e serão, utilizados.

\subsection{Tecnologias}
A seguir estão as tecnologias utilizadas, suas características principais e, assim, porque foram escolhidas. A finalidade principal desse conjunto é escrever a aplicação de forma rápida e eficiente, concentrando toda a energia no desenvolvimento e aplicação da lógica, e, logo, poupando tempo em funcionalidades básicas.

\subsubsection{Django}
É um \gls{framework} \textit{web} \textit{\gls{open source}} e de alto nível, desenvolvido em Python, que se baseia no padrão \ac{mtv}, apresentando semelhança com o \ac{mvc}. Assim, segue o princípio \ac{dry}\footnote{Permite que as aplicações sejam desenvolvidas com a maior quantidade de aproveitamento de código possível.}, é moderadamente opinativo \footnote{Flexibilidade que o \gls{framework} dá aos desenvolvedores à resolução dos problemas. Opinativo, já possui uma maneira correta de resolvê-los, sem margens; não-opinativo, não possui essas regras e deixa livre para resolvê-los como quiser. Django equilíbrio entre soluções prontas e arquitetura desacoplada com liberdade na resolução de erros. }e apresenta suporte para erros comuns de segurança. Além desses benefícios, foi escolhido devido a sua aplicação em grandes empresas (como \gls{mozilla} e \gls{pinterest}) e, principalmente, no \ac{suap} do \ac{ifsp}, o que permite manter o padrão de tecnologias no Instituto.
\cite{lucas_2021} \cite{andrade_2019}

\paragraph{\ac{mtv}}
Derivação da arquitetura de software \ac{mvc}, de três camadas, altera a nomenclatura e a relação entre os arquivos. O Model permanece o mesmo, como um canal de conexão entre os tipos de dados e como serão armazenados no Banco de Dados, e a exibição ao ter requisição à View. Essa é responsável, então, pelo gerenciamento das requisições e a lógica de negócio, com a formatação dos dados enviados pelo Model. Por fim, o Template é a interação com o usuário, através de uma exibição estática ou inserção de sintaxe de conteúdo dinâmico, com a renderização dos dados entregues pela View.
\cite{silva_2020}
\
\subsubsection{Python}
É uma linguagem de programação \textit{\gls{open source}} e de alto nível, interpretada em scripts e orientada a objetos, que apresenta tipagem dinâmica\footnote{Tipo do dado é determinado no tempo de execução, de acordo com o valor do dado, não a partir da sua variável.} forte. Logo, prioriza a agilidade por meio de sua fácil compreensão, sintaxe menor e simplificada, sem muitas exigências gramaticais. E é por isso que foi escolhida, uma ótima opção que supriu de forma excelente a necessidade de uma aprendizagem rápida e fácil codificação nos dispositivos do \ac{ifsp}, além de poder ser facilmente integrada a outras linguagens de programação populares, caso seja necessário no decorrer do projeto.
\cite{amazon_2023} \cite{melo_2021}

\subsubsection{AJAX}
O \ac{ajax} é uma técnica de desenvolvimento \textit{web}, caracterizada pela criação de aplicações interativas através de requisições ao servidor. Uma junção das funcionalidades do \ac{js} com a troca dos dados, armazenados e transmitidos, nesse caso, pelo \ac{json} (mais próximo do \ac{js}). Foi escolhido justamente por servir como um canal de comunicação independente entre o cliente e o servidor.
\cite{andrei_2019} \cite{carvalho_2007}

\subsubsection{JavaScript}
É uma linguagem de programação de alto nível e interpretada em scripts, com recursos de \ac{oo} e \ac{api}, que apresenta tipagem dinâmica. Assim, por meio de um funcionamento assíncrono \footnote{A programação assíncrona é uma técnica na qual o programa inicia uma tarefa e ainda é capaz de executar simultaneamente outros eventos, ao invés de bloquear processos para esperar o término da execução.}, usa trechos dos códigos HTML para renderizar funções que proporcionem uma interação dinâmica local com o conteúdo da página. Foi escolhida para, em conjunto com o \ac{ajax}, proporcionar essa dinamicidade em tempo real, recarregamento automático.
\cite{mozilla_2023} \cite{melo_2021}

\subsubsection{HTML}
O \ac{html} é uma estrutura responsável pela exibição dos dados no navegador \textit{web}, caracterizado por seus elementos hierarquizados e sua marcação que abriga elementos como tags. Na aplicação ADA, é utilizada nos templates, explicados no parágrafo 2.4.1.1.1 .
\cite{mozilla_2023b}

\subsubsection{CSS}
O \ac{css} é uma linguagem de marcação, responsável pela estilização de elementos \ac{html}. Foi escolhido a fim de ajudar na formatação dos templates em detalhes específicos que, por vezes, não são compreendidos pelo framework, pois esse é mais genérico.
\cite{totvs_2020}

\subsubsection{Bootstrap}
É um \gls{framework} \textit{front-end}, logo, voltado à estilização, e \textit{\gls{open source}}. Foi escolhido devido à agilidade no desenvolvimento da página para o usuário, característica pelos frameworks, e, principalmente, devido à responsividade proporcionada.
\cite{andrei_2019} \cite{lima_2021}

\subsubsection{SQLite3}
\ac{sgbd}, o SQLite é uma biblioteca em linguagem C, \textit{\gls{open source}}, acoplada ao banco de dados \ac{sql}. Escolhido porque entrega o banco em conjunto com a aplicação (é embutido), sem a necessidade de um servidor, já que optamos por realizar a arquitetura \ac{mtv} em vez de Cliente/Servidor.
\cite{silva_2007} \cite{carlos_2019}

 \subsection{Hospedagem}
 A fim de ter uma instância em nuvem e conseguir fazer a hospedagem do site, foi utilizada a \ac{aws}. É uma plataforma que disponibiliza diversos serviços de computação em uma rede de servidores remotos. Assim, é possível criar instâncias de máquinas com sistema operacional Windows ou Linux, de modo que a aplicação funcione constantemente, sem necessitar que um computador pessoal fique ligado. 

 Somente com ela já é disponibilizada a aplicação na \textit{web}. Todavia, o acesso é difícil, pois aparecerá somente o endereço IPv4 público da máquina virtual criada. Para resolvê-lo, foi comprado o domínio \url{https://mottarios.cloud/} no \textit{website} Hostinger.

 \subsection{Criptografia}
A criptografia, para fornecer uma aplicação segura, foi configurada seguindo o protocolo \ac{https}, 

De forma a verificar seu devido funcionamento, foi utilizada a certificadora \textit{Let’s Encrypt}. Assim, teve uma confirmação de que há controle sobre o domínio citado anteriormente, e, com isso, foi gerado um certificado \ac{ssl}/\ac{tls} para ele. Após adicionar o certificado, a aplicação atingiu nota A no \textit{SSL Labs}. Para mais informações: \url{https://www.ssllabs.com/ssltest/analyze.html?d=mottarios.cloud}
\
\begin{figure}[h]
    \centering
    \includegraphics[width=1.00\textwidth]{anexos/DiagramaDeArquitetura/Certificado_NotaA.jpeg}
    \caption{Certificado SSL/TLS}
    \label{fig:figura1} 
\end{figure}

\subsection{Ferramentas}

\subsubsection{Controle de Versão}
É utilizado o \ac{svn}, uma ferramenta que armazena projetos e todas suas versões em um servidor centralizado. Escolhido devido à padronização dos projetos e devido à capacidade de armazená-los de forma segura. Igualmente, é utilizado o \gls{github}, de forma secundária, devido à familiaridade e à melhor organização, principalmente tratando-se do uso de \textit{branchs}.

\subsubsection{Documentação}
É utilizado o Overleaf, um editor e compilador \textit{online} de \gls{latex}. A fim de seguir a padronização dos projetos anteriores e para uma produção mais dinâmica dos documentos \gls{latex}, já que permite o compartilhamento dos arquivos entre outras pessoas e a edição simultânea. 

\subsubsection{Programação}
É utilizado o \ac{vscode}, um editor de código-fonte criado pela Microsoft. Foi escolhido devido a sua dinamicidade para codificação e a familiaridade que os membros da equipe já possuem com a ferramenta. 

Ademais, também é utilizado o codespace, que, por sua vez, é um ambiente de desenvolvimento hospedado em nuvem, promovido pela plataforma GitHub. Assim, é possível programar através do navegador, sem precisar de uma aplicação instalada na máquina, facilitando a programação no \ac{ifsp}. 

\subsection{Diagrama de Arquitetura}

\begin{figure}[h]
    \centering
    \includegraphics[width=0.96\textwidth]{anexos/DiagramaDeArquitetura/DiagramaDeArquitetura.jpeg}
    \caption{Diagrama de arquitetura}
    \label{fig:figura1} 
\end{figure}

%  LISTA
% \begin{itemize}
%   \item List entries start with the \verb|\item| command.
% \end{itemize}

%---------------------------------------------------------------------------------------







\input{textos-conclusao}

% ----------------------------------------------------------
% Finaliza a parte no bookmark do PDF
% para que se inicie o bookmark na raiz
% e adiciona espaço de parte no Sumário
% ----------------------------------------------------------
\phantompart

% ----------------------------------------------------------
% ELEMENTOS PÓS-TEXTUAIS
% ----------------------------------------------------------
\postextual

% ----------------------------------------------------------
% Referências bibliográficas
% ----------------------------------------------------------
\printbibliography

\input{pos-glossario.tex}

%---------------------------------------------------------------------
\phantompart

\end{document}