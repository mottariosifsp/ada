
% Definições para glossario

% ATENCAO o SHARELATEX GERA O GLOSSARIO/LISTAS DE SIGLAS SOMENTE UMA VEZ
% CASO SEJA FEITA ALGUMA ALTERAÇÃO NA LISTA DE SIGLAS OU GLOSSARIO É NECESSARIO UTILIZAR A OPÇÃO :
% "Clear Cached Files" DISPONIVEL NA VISUALIZAÇÃO DOS LOGS 
% ---
% https://www.sharelatex.com/learn/Glossaries

% Normalmente somente as palavras referenciadas são impressas no glossário, portanto é necessário referenciar utilizando :
% \gls{identificação}            
% \Gls{identificação}            
% \glspl{identificação}            
% \Glspl{identificação} 

\newglossaryentry{admin}{
                name={Admin},
                description={Administradores, \gls{enum} que representa os funcionários que receberão essa funcionalidade naquele semestre/ano.} } 

\newglossaryentry{angular}{
                name={Angular},
                description={Plataforma de aplicações web de código-fonte aberto e \gls{frontend} baseado em TypeScript.} }

\newglossaryentry{backend}{
                name={backend},
                description={Parte da aplicação que gerencia as conexões e a interação com o banco de dados.} }

\newglossaryentry{codelab}{
                name={CodeLab},
                description={Projeto de ensino equiapado pelo Instituto Federal de São Paulo que visa práticas de um ambiente profissional de desenvolvimento softare.} }

\newglossaryentry{enum}{
                name={enum},
                description={Tipo de classe especial que permite o armazenamento de um grupo de valores constantes.} }

\newglossaryentry{excel}{
                name={Excel},
                description={Editor de outras planilhas produzido pela empresa Microsoft.} }

\newglossaryentry{frontend}{
                name={frontend},
                description={Parte gráfica da aplicação que o usuário pode interagir.} }

\newglossaryentry{git}{
                name={Git},
                description={Sistema utilizado principalmente por programadores para hospedar códigos e arquivos versionados pelo Git.} }

\newglossaryentry{github}{
                name={GitHub},
                description={Sistema utilizado para hospedar códigos e arquivos versionados pelo \gls{git}. Disponível no endereço \url{<https://github.com/>.}} }

\newglossaryentry{globo}{
                name=Globo,
                description={Rede de televisão comercial brasileira aberta.} }

\newglossaryentry{feedback}{
                name={feedback},
                plural={feedbacks},
                description={Avaliação de uma ação ou de uma pessoa que leva em consideração uma série de fatores sobre o processo almejado; pode ser positiva ou negativa.} }

\newglossaryentry{framework}{
                name={framework},
                plural={frameworks},
                description={Junção de códigos, como bibliotecas, que traz como resultado final uma funcionalidade genérica.} }

\newglossaryentry{jsonwt}{
                name={JSON Web Token},
                description={Padronização que busca armazenar e transmitir objetos do tipo JSON de forma segura.} }

\newglossaryentry{login}{
                name={login},
                description={Processo de acesso em um sistema através dos dados de identificação de um usuário.} }

\newglossaryentry{microsoft}{
                name={Microsoft},
                description={Uma das maiores empresas de tecnologia criada por Bill Gates e Paul Allen.} }

\newglossaryentry{mysql}{
                name=MySQL,
                description={Sistema de gerenciamento de banco de dados que trabalha com a linguagem SQL como interface.} }

\newglossaryentry{permuta}{
                name={permuta},
                plural={permutas},
                description={Trocas; nesse caso, de aulas.} }

\newglossaryentry{permutação}{
                name={permutação},
                description={Processo de troca; nesse caso, entre dois docentes, seguindo todos critérios de regramento.} }

\newglossaryentry{professor}{
                name={Professor},
                description={Professor, \gls{enum} que representa o docente no sistema.} }

\newglossaryentry{springboot}{
                name={Spring Boot},
                description={\textit{\gls{framework}} desenvolvido para a plataforma Java baseado nos padrões de projetos.} }

\newglossaryentry{superAdmin}{
                name={SuperAdmin},
                description={Administrador Superior, \gls{enum} que representa os funcionários que receberão essa funcionalidade naquele semestre/ano.} }

\newglossaryentry{svn} {
                name={SVN},
                description={Tem como objetivo fazer o controle de versão de arquivos, inicialmente tinha o propósito de ser um substituto melhorado do Sistema de Versões Simultâneas.} }

\newglossaryentry{typescript}{
                name={TypeScript},
                description={Linguagem de programação tipada que se baseia em JavaScript.} }